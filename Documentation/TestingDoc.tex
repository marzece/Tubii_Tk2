\title{TUBII: A Field of Dreams}
\documentclass[11pt,a4paper]{article}
\usepackage[utf8]{inputenc}
\usepackage{amsmath}
\usepackage{amsfonts}
\usepackage{amssymb}
\usepackage{hyperref}
\usepackage{authblk}
\usepackage{verbatim}


\title{Tubii: The Proving Ground}
\author{Eric Marzec}

\begin{document}
\maketitle
\begin{abstract}

\end{abstract}
\begin{itemize}
\item Make Testing Document
\end{itemize}
\section{Powers}
\begin{itemize}
\item Use multimeter check that there are no shorts between the various powers.
(VCC,GND,V3P3,VEE,VTT,VCCIO,V15,V15M)
\item Plug in the board to each power if they're availible
\item Test regulators (note the VCCIO regulators output should be 0V until the 
enable pin is used) (Or maybe on second thought the output will be 3.3V when b/c the
enable pin is disconnected,I'm not sure, but you can test that the output turns on/ off when the enable pin is low/high)
\end{itemize}
\section{MicroZed}
\begin{itemize}
\item Stuff the lv06a that will control the VCCIO regulator. Then stuff and plug in the MicroZed (MZ)
\item Make sure the MZ turns on (LEDs light up and all)
\item Make sure the VCCIO banks get 3.3V
\item Check that the various lv07a/lvc07a's all work and that outputs from the MZ
make it through them with decent gain (ie 3.3V signals go to 5V signals)(Note the shift register CLK/Data line may be an excpetion here, more on that later)
\item MZ Happy light is on when MZ is plugged in and turned on
\item MZ Hppy light is off when MZ is unplogged and or off
\end{itemize}
\section{Multiplexer}
\begin{itemize}
\item Check that each the multiplexer turns on/off appropriately
\item Check that each line can be addressed as expected
\end{itemize}
\section {ControlRegister}
\begin{itemize}
\item The 74hct164 can be loaded with arbitrary 8bits.
\item These 8bits don't showup at the  various parts they feed into until the data
ready line is strobed.
\item The 8bits can be read back in in a non-destructive manner
\end{itemize}
\section{GlobalTriggering and GT Delays}
\begin{itemize}
\item Confirm that the GT shows up on TUBii and looks alright (squarish)
\item Check that it gets delayed (at all) by the two DS1023s in GT\_Delays section
\item Check that these delays can be changed by loading their shift registers.
\item Check that the daisy chainging of their shift registers works
\item Check that the GT gets to the MZ
\item Check that the MZ is able to count GTs and keep a running GTID
\item The Microzed can choose between DDGT and LO\_MTCD
\item Check the all pulses (GT,DGT,DDGT,LO*) are down-going or up-going as they 
should be. NOTE TO SELF, FIGURE OUT HOW THEY SHOULD BE GOING
\end{itemize}
\section{Clocks}
\subsection{Default Clock Select}
\begin{itemize}
\item Some surgery is needed to get the clock in place. Check that a 200MHZ signal comes out of it at all.
\item Check the the LVPECL pull-down is working well
\item Check that the 200MZ signal gets divided to 100MHZ
\item Clock divider's reset button is working
\item Clock divider reset signal from MZ is working
\item Quality of external TUB clock is good
\item One clock becomes Default and the other becomes backup
\item The clock that is default can be switched with backup and vice versa
\end{itemize}
\subsection{Fault Detection}
\begin{itemize}
\item The DefaultClock (DefCLK) signal's frequency gets divided and the various jumpers can pick between frequencies.
\item No DefCLK signal makes output of the HCT123 changed.
\item Some noticable change in the system happens when the DefCLK output is missing
\item You can count how many clock pulses get missed. (ie you get 100mhz pulses while the DefCLK is gone and they go away when DefCLK shows up again)
\item The mc10e016 emits a signal if many clock pulses are missed
\item The MC10e016's shift register can be loaded and this allows you to pick
how many pulses until TC does something 
\item The output of that fucking rats nest that hangs off of TC makes at least some sense, good fucking luck buddy.
\item The MZ gets told when the clocks should be changed
\end{itemize}
\subsection{Change Clocks}
\begin{itemize}
\item When physical switch is thrown one way the clock at output is exactly the same
as back up clock
\item When thrown the other way the output clock is the DefCLK unless DefCLK 
ChangeCLK signal is high.
\item The various LEDs light up in a way that makes some sense/is useful
\end{itemize}
\section{Ecal Control}
\begin{itemize}
\item The control register outputs an ECAL\_ACTIVE signal that can be actively
changed by the MZ.
\item The LED corresponding the the ECAL\_ACTIVE signal works
\item When ECAL\_ACTIVE is high the output is GT
\item When ECAL\_ACTIVE is low the output is EXT\_PED\_IN
\end{itemize}
\section{ELLIE}


 
\section{CAEN Interface}
\subsection{CAEN Digital}
\subsection{CAEN Analog}
\begin{itemize}
\item This will never work, give up
\end{itemize}
 
\section{Baseline Monitoring}
\begin{itemize}
\item I'm not exactly sure how to test this...figure it out
\end{itemize}



\section{MTCA\_MIMIC}
\begin{itemize}
\item DAC can be set by the MZ to output a sensible volatage value
\item The stupid pot can be tuned to output equally sensible voltage values
\item Comparator outputs signal appropriate signal when Analog Pulse is over/under
 DAC threshold
 \item Comparatoroutputs signal appropriate signal when Analog Pulse is over/under
 POT threshold


\item GT DGT, and LO* all show up as expected.
\item The DGT\_Gate signal makes sense.
\end{itemize}
\subsection{Trigger Logic}
\begin{itemize}


\item Comparator outputs high when a signal goes over the DAC value AND the 
physical switch is thrown such that positive going pulses are selected

\item Comparator outputs high when a signal goes over the POT value AND the 
physical switch is thrown such that positive going pulses are selected

\item Comparator outputs high when a signal goes under the DAC value AND the 
physical switch is thrown such that negative going pulses are selected

\item Comparator outputs high when a signal goes under the POT value AND the 
physical switch is thrown such that negaitve going pulses are selected
\end{itemize}
\section{General Utilities}
\subsection{Generic Delays}
\begin{itemize}
\item Emits a TTL that can be delay by a tuneable amount that roughly matches the input signal
\item Blinks an LED that matches the delay
\end{itemize}
\subsection{Generic Pulser}
\begin{itemize}
\item Emits a TTL pulse at a frequency that can be chosen by the user.
\end{itemize}
\subsection{Pulse Inverter}
\begin{itemize}
\item Analog pulses can be changed from upward going to downward going
\item Analog pulses can be changed from downward going to upward going
\end{itemize}

\subsection{Ribbon Delay}
\begin{itemize}
\item ECL pulses can be delayed by an amount that makes physical sense 
(i.e. a meter long cable leads to a few ns of delay)
\end{itemize}

\subsection{Pulse Scaler}
\begin{itemize}
\item The MZ can output signals that increment the the display
\item The display can show the frequency (in Hz) of signals
\item lead zero blanking can be turned on/off by control register
\end{itemize}
\subsection{Translation}
\begin{itemize}
\item TTL pulses go to ECL pulses
\item ECL pulses go to TTL pulses
\item LVDS pulses go to ECL pulses
\item ECL pulses go to LVDS pulses
\item NIM pulses go to ECL pulses
\item ECL pulses go to NIM pulses
\end{itemize}
\section{Speaker}
\begin{itemize}
\item Speaker clicks when told to do so by MZ
\item Speaker is loud enough
\item Speakers loudness can kinda be tweaked by speaker pot
\item Speaker outputs a signal that can be recorded by a computer
\end{itemize}

\section{Shift Registers}
Below is a list of all the shift registers on the TUBII. The LE MUX is the chip described in \ref{SR_Addr}. More specifically it is an hct238.  For exact information on how to address each pin refer to the hct238's datasheet.
\subsection{Control Register}
$\bullet$ 8 bits. Serial in, parallel out\\
$\bullet$ Connected to pin 13 of LE MUX\\
$\bullet$ Designed to be programmed once at start up.\\
$\bullet$ Output gated by flip flops. Toggle $CNTRL\_RDY$ once programming is finished\\
$\bullet$ Each output acts as an input for the Read Control Register\\
\subsection{Read Control Register}
$\bullet$ 8 bits. Parallel in, serial out \\
$\bullet$ Used to read state of the Control Register non-destructively \\
$\bullet$ MZ toggles reads it serially by toggling Read\_Enable pin (pin 1) each time it wants to see the next bit.\\
$\bullet$ The register is circular. So reads should be done 8 at a time.\\
\subsection{CAEN Register}
$\bullet$ 16 bits. Serial in, parallel out\\
$\bullet$ Connected to pin 15 of LE MUX \\
$\bullet$ The bits 0-3 select between analog pulses for the CAEN\\
$\bullet$ The 4-7 bits do nothing\\
$\bullet$ The bits 8-15 choose if analog pulses go to the CAEN or the scope outputs\\
$\bullet$ Bits 8-15 are gated by flip-flops. Toggle $CAEN\_RDY$ once programming is finished\\
\subsection{GT Delays Register}
$\bullet$ 16 bits. Serial in. No output\\
$\bullet$ Connected to pin 14 of LE MUX\\
$\bullet$ The data sets the delay length for the two delays\\
$\bullet$ Bits 0-7 set the time between GT and DGT\\
$\bullet$ Bits 8-15 set the time between GT and DDGT (LO)\\
$\bullet$ Bits 0-7 are for a DS1023200 and bits 8-15 are for a DS1023500. See their respective datasheets for programming information.\\
\subsection{Generic Async Pulse Register}
$\bullet$ 8 bits. Serial in. No output\\
$\bullet$ Connected to pin 7 of LE MUX \\
$\bullet$ The data sets the delay length for the generic pulser. \\
$\bullet$ The delay length should not exceed the time between pulses. I.e it should not try to delay a 1GHz pulse by 10ns.\\
$\bullet$ The delay chip is a ds102350. See it's datasheet for programming\\ information.\\
\subsection{Generic Async Delay Register}
$\bullet$ 8 bits. Serial in. No output \\
$\bullet$ Connected to pin 9 of LE MUX \\
$\bullet$ The data sets the delay length for the generic delay. \\
$\bullet$ The delay length should not exceed the time between pulses. I.e it should not try to delay a 1GHz pulse by 10ns.\\
$\bullet$ The delay chip is a ds102350. See it's datasheet for programming\\ information.\\
\subsection{Comparator DAC Register}
$\bullet$ 12 bits. Serial in. No output \\
$\bullet$ Connect to pin 11 of LE MUX \\
$\bullet$ The data sets the output voltage value for a DAC which acts as a threshold for one of the Comparators. \\
$\bullet$ The DAC is a AD7243. See datasheet for programming information.\\
\subsection{Missed Clock Count Register}
$\bullet$ 8 bits. Serial in, parallel out \\
$\bullet$ Connected to pin 10 of LE MUX \\
$\bullet$ Data is gated. Toggle $TubiiTime\_Data\_Time$ once programming is finished.\\
$\bullet$ Outputs the value for how many missed counts are allowed before clock switch over \\

\end{document}
