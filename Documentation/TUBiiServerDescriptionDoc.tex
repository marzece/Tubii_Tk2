\documentclass[12pt,a4paper]{article}
\usepackage[utf8]{inputenc}
\author{Eric Marzec}
\title{TUBii Server API Description}
\begin{document}
\maketitle
Described here is the set of commands I think would be sufficient and convenient for the server runing on TUBii to implement.
\begin{itemize}
\item
SetGTDelays(uint8 LO\_Delay, uint8 DGT\_Delay)\\
\{\\
muxer 3\\
muxenable 1\\
loadshift LO\_Delay\\
loadshift DGT\_Delay\\
muxenable 0

self.LO\_Delay = LO\_Delay
self.DGT\_Delay = DGT\_Delay

return ErrorDidHappen?\\
\}
\item 
GetLODelay()
\{\\
  return self.LO\_Delay\\
\}
\item
GetDGTDelay()
\{\\
  return self.DGT\_Delay\\
\}
\item
SetCaenWords( uint8 GainPathWord, uint8 ChannelSelectWord)\\
\{\\
muxer 1 \\
muxenable 1 \\
loadshift GainPathWord \\
loadshift ChannelSelectWord \\
muxennable 0\\
dataready 6\\
dataready 4\\

self.GainPathWord = GainPathWord\\
self.ChannelSelectWord = ChannelSelectWord\\

	return ErrorDidHappen?\\
\}\\
\item 
GetCAENGainPathWord()\\
\{\\
  return self.GainPathWord\\
\}\\
\item
GetCAENChannelSelectWord()\\
\{\\
  return self.ChannelSelectWord\\
\}\\
\item
SetControlReg(uint8 ControlRegWord)\\
\{\\
muxer 0\\
muxenable 1\\
loadshift ControlRegWord\\
muxenable 0\\
dataready 5\\
dataready 4\\

self.ControlRegWord = ControlRegWord\\
(Perhaps here is where we would like to use the ReadShift command to 
confirm that things actually worked)\\
return ErrorDidHappen?\\
\}\\
\item
GetControlReg()\\
\{\\
(Maybe we want to use the readshift command here as well)\\
return self.ControlRegWord\\
\}\\
\item
SetDACThreshold(uint16 DACWord)
\{\\
muxer 2\\
muxenable 1\\
muxenable 0\\
(The DAC word is 16 bits (12 actually) but loadshift currently writes 8 at a 
time. You'll have to figure out how you wanna deal with this.
Hopefully that won't be too difficult)\\

loadshift DACWord1\\
loadshift DACWord2\\
dataready 0\\
dataready 4\\

self.DACWord = DACWord\\

return ErrorDidHappen?\\
\}\\
\item
GetDACThreshold()\\
\{\\
return self.DACWord\\
\}\\
\item
SetAllowableClockMisses(uint8 NMisses)\\
\{\\
(I've actually only ever changed this on the TUBii once. 
So I don't actually know the commands. I'll have to figure this one out later)\\

self.AllowableMisses = self.AllowableMisses

return ErrorDidHappen?
\}\\
\item
smelliePulser(rate, width,nPulses)\\
\{\\
(Whatever this function already does)\\
self.smellieRate = rate\\
self.smellieWidth = width\\
self.smellieNPulses = nPulses\\

return ErrorDidHappen?\\
\}\\
\item
GetSmellieRate()\\
\{\\
return self.smellieRate\\
\}\\
\item
GetSmellieWidth()\\
\{\\
return self.smellieWidth\\
\}\\
\item
GetSmellieNPulses()\\
\{\\
return self.smellieNPulses\\
\}\\
\item
smellieDelay(length)\\
\{\\
(Whatever this function already does)\\
self.smellieDelayLength = Length\\
\}\\
\item
GetSmellieDelay()\\
\{\\
return self.smellieDelayLength
\}\\
\item
Repeat all the smellie functions for tellie\\
\item
Repeat all the smellie functions for Generic\\
\item
clockReset()\\
\{\\
sets the clock reset pin high than low\\
This is somewhat different than the current version of clockReset which
requires an arguement)\\

return ErrorDidHappen?
\}\\
\item
clockStatus()\\
\{\\
(same as what it alredy does except returns value not in datastream)\\
\}\\
\item
countMask(countMask)\\
\{\\
(same as what it alredy does)\\
self.countMask = countMask\\
return ErrorDidHappen?\\
\}\\
\item
GetCountMask()\\
\{\\
return self.countMask\\
\}\\
\item
Same functions that count has except for the speaker\\
\item
Getters/Setters for trigger mask 
\item
Something to do with setting combo/burst/prescale?\\
You probably know better about how necessary getters/setters for that are\\
\end{itemize}

\end{document}
